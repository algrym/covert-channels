\documentclass[letterpaper,10pt,draft]{article}

\usepackage{verbatim} % for verbatim blocks
\usepackage{lastpage} % for lastpage counter

\usepackage{fancyhdr} % for \pagestyle{fancy} and \{lcr}{head,foot}
\pagestyle{fancy}

% CVS definitions
\def\RCS$#1: #2 ${\expandafter\def\csname RCS#1\endcsname{#2}}
\RCS$Id: covert.tex,v 1.16 2006/08/09 19:36:46 ajw Exp $ % or any RCS keyword

% define author information
\def\AJWName{A. J. Wright}
\def\AJWEmail{\texttt{<ajw@utk.edu>}}
\def\AJWSubject{Covert Channels}

% setup document information
\title{\AJWSubject\ using TCP/IP Header Identification Fields}
\author{\AJWName \\ Information Security Office \\ University of Tennessee \\ \texttt{\AJWEmail}}
\date{\small{\RCSId}}

% Define the content of the header
%\lhead{\textbf{\AJWSubject}}
%\rhead{\textbf{\AJWName}}

% define the content of the footer
\cfoot{Page {\thepage} of \pageref{LastPage} \\ \tiny{\RCSId}}

% setup PDF information
\usepackage[pdftex]{color,graphicx,hyperref}

\begin{document}

\maketitle

%\tableofcontents
%\newpage

\section*{Abstract}

Covert channels are used to secretly convey information from one place
to another.  This document presents two means for covert network
communication using fields in the TCP and IP headers in an novel
manner.

\section{Background}

Computer criminals often attack personal, business, government, or
educational systems that contain valuable data.  This data can include
customer names and credit card numbers, confidential business
information, compromising photographs, or government classified
information.  Once access to a system has been obtained, the attacker
frequently is required to copy the data over the Internet to a safe
location before it can be further utilized.

It is important for the attacker to remain hidden while assaulting a
system and retrieving its data.  If detected, there are any number of
defenses and obstacles which can be erected by the system owner to
protect a computer and its data, even after the system has been
compromised.  In many cases, the attacker is accessing the system
illegally.  Discovery is the prelude to collecting evidence that can
be used to pursue litigious forms of punishment and monetary damage.

%Larger databases and documents may be more valuable due to their
%containing a larger number of records or a greater amount of
%information.  A list of ten million credit card numbers has more worth
%than a list of only a hundred. Unfortunately for the attacker, the
%larger a document or database is, the more time it will take to
%transmit across the network and the more chances it has of alerting
%those responsible for protecting the victimized systems and their data.

If the original system compromise does not alert security staff at the
victim organization, the bandwidth use and extra traffic as the data
is copied across the network may.  Intrusion detection systems
(IDSs), intrusion prevention systems (IPSs), and other network
monitors can be configured to detect unusual outgoing traffic from any
given system.  In particular, many firewalls can block traffic to or
from a specific host or network based on a packet's source address, source
port, destination address, and destination port.

To copy the valuable data and still avoid blockage or detection, the
attacker must use a ``covert channel'' (CC), which is a means of
transmitting information surreptitously.  In some cases, this is
accomplished by making the attacker's traffic appears to be as
indistinguishable as possible from routine network traffic. IDSs,
IPSs, and firewalls should allow the packets associated with the CC to
pass unmolested, ``believing'' they are legitimate traffic.  The goal
is not merely to obscure the existence of a transmission, but by its
obscurity to prevent detection of the attacker and the system
compromise.  By avoiding detection, the attacker can maintain control
over the compromised system for as great a time as possible.

Following information gathering on the subject of covert channels, it
was decided that a well-informed and serious attacker could use an
advanced covert channel to bypass network detection and protection
system.  In order to bring awareness to information leaks in general
and to help decrease the effectiveness of this type of attack,
proactive examination of this area should be performed to assess means
by which an attacker might use a CC to leak information from a network
entity and to investigate how it could be detected.

\subsection{Goals, Scope, and Constraints}

The goal of this investigation is to implement a proof-of-concept
covert channel that is capable of circumventing traditional means of
detection and with the new findings, to examine ways to detect that
proof-of-concept channel.

This document discusses the scope and background for the
investigation, describe the proof-of-concept software, catalog means
for detecting and defending against this type of covert channel, and
detail potential areas for future effort.

\subsubsection{Scope and Assumptions}

For the scope of this inquiry, there are some assumptions made
regarding so-called ``traditional means of detection'' which are
generally accurate in the Information Security Office at The
University of Tennessee.\footnote{The exception being cases in which
integrity of the University's network is threatened.}  As the members
of that group bring together representative views of industry
training and standards, it is extrapolated that these assumptions are
true to some degree in other security organizations.

First, more suspicious packets are available to any number of security
analysts than they have time to personally and completely analyze.
(This is a safe assumption: no analyst has time to check all packets
even from their own workstation.)  To leverage their own time, IDSs,
IPSs, and firewalls will be used to block \textit{known} undesirable
traffic and help identify suspicious traffic for later analysis.

To attack a security analyst's packet review process, a CC should not
use traditional locations in a network packet--such as the payload
area--to transmit data.  Instead, certain identification and
enumerated sequence fields will be used.

Second, IDSs and IPSs rely on signatures to identify and (in the case of
IPSs) block suspicious or undesirable traffic.  Firewalls do a similar
task, examining a combination of header fields.  While these systems
maintain limited state regarding traffic flows, there will always be
more information supplied and suggested by visible network traffic
than there will be available computing resources.

The attack against these systems will be three-part. First, the
use of packet identification fields to utilize a scope
wider than what can be stored by an IDS or IPS.  Second, to avoid
detection by signature, no static strings can be used to identify a
single packet as being associated with a CC.  Third, to avoid blockage
by firewall, existing (and assumed successful) traffic will be sniffed
off the wire, modified, and retransmitted.  

For complete undetectability,\footnote{It is obviously na\"ive to say
that a channel is ``undetectable'' when new means for discovery are
commonly being developed.  Here the word is used instead of the more
cumbersome and correct phrase ``difficult to be detected.''} there
should be no commonality between any series of packets in the entire
stream.  While the proof-of-concept seeks to diminish the stream
commonality as much as possible, a true and complete implementation of
this idea is more complex than is addressable by the scope of this
investigation.

There are a number of areas which are declared to be desirable for
practical use, but are outside the scope of this document and the
underlying inquiry.

\begin{itemize}
\item \textbf{Encryption:} All network traffic of a sensitive nature
should be encrypted.  Because the attacker is attempted to conceal
information--specifically their existence on the network--the covert
channel traffic should be encrypted.  Pre-encryption of data or
addition of one of the large number of strong encryption packages
available is an exercise for the future.

\item \textbf{Bi-directional communication:} The methods for
communication that are described later are strictly one-way.  Some
suggestions are made for potential means of adding a return channel.

\item \textbf{Stream integrity and reliability:} Without a return
channel, guaranteeing the stream integrity becomes difficult or
impossible.  Suggestions are made in later sections for use of erasure
codes to increase the quality of the overall stream.

\item \textbf{Functionality:} One concern of the administration was
creating a tool so powerful that it was introducing software that made
a new attack simpler, effectively ``selling arms to the enemy.''  The
software written in the course of this investigation deliberately
avoids providing a complete communications solution.  It is meant to
provide a platform for demonstration of the ideas detailed in this
document, specifically: one-way transmission of a document between two
program instances on the same unswitched network.  By omitting some
necessary components, it is hoped that these tools can be made useless
to ``script kiddies'' and other malicious attackers without
significant extra effort.

\end{itemize}

\subsubsection{Constraints}

To be successful, a CC must take advantage of several weaknesses in
traditional network attack detection.  

Criteria were developed that discuss standard detection methods
and what a covert channel can and cannot do in order to
remain undetectable.

%To best avoid detection, the CC must use what are typically non-data
%fields to store data.  Instead of using the payload areas of an
%Internet protocol (IP) packet, header areas must be used.  Existing
%research--specifically Loki--uses the Internet control message
%protocol (ICMP) data area to transmit information.  Unfortunately,
%many firewalls block ICMP messages to prevent information leaks
%regarding network infrastructure.  As a bonus just by blocking ICMP,
%the firewalls can also end up blocking Loki traffic rendering it
%unusable.

\begin{enumerate}

\item \textbf{The covert channel must not interfere with normal
delivery of its own packets.} Network packet header
fields used to deliver or route a packet from one point to another
are off-limits for data transmission use.

\item \textbf{There should be no single or combination static
signature that marks a packet as part of a covert channel.}  Traffic
signatures are traditional means by which an IDSs or IPSs is able to
identify and block undesirable traffic.  If every packet in the
transmission contains a known value or string, a signature can be
created allowing for easy detection and removal.  Unfortunately, this
makes it difficult for the CC software to recognize packets that are
meant for it.

\item \textbf{The covert channel must only use packet fields where
    blocking any single value within the scope of that field would
    interfere with normal network traffic.}  IP options and other
    ``additional but not required'' components are not available for
    use, specifically because they are not \textit{required.}  These
    optional fields can be blocked without affecting the overall data
    stream.  Instead, only fields that represent packet identification
    numbers can be used: fragmentation offset, IP identification
    field, TCP sequence and acknowledgment numbers, and similar
    elements can be used.  In any of those cases, blocking a
    particular value will cause normal use of the involved protocols
    to malfunction.

\item \textbf{The covert channel must utilize existing network traffic
in a least-impact manner.}  To avoid detection, the best time to send
a covert packet is when it can blend in with other traffic that is
traversing the network.  Covert traffic should not cause alerts on any
system that it is associated with, and if possible it should cause no
alerts whatsoever.
\end{enumerate}

%To best meet these specifications, packet headers in the Internet
%protocol (IP) network suite were examined as a location for payload
%data transmission.  It was decided that the ... 

\section{Initial Proof of Concept}

For the initial proof-of-concept implementation (POC1), several major
decisions had to be made.  This version only works across a single
subnet space.  It becomes especially useful in obfuscating the sender
and receiver on a single-segment or super-netted network, such as a
large wireless network.

The POC1 is a console-mode application developed in C on Mac OS X 10.4
(a BSD variant) using \textit{gcc} as provided in Apple's Xcode
development
suite.\footnote{\texttt{http://www.apple.com/macosx/features/xcode/}}
Packets are captured off the wire using the near-ubiquitous library
\textit{libpcap}\footnote{\texttt{http://www.tcpdump.org/}} and are
transmitted using the kernel-interface routines in
\textit{libdnet}.\footnote{\texttt{http://libdnet.sourceforge.net/}}
Both of these libraries were written to allow abstract access to
low-level network functions in a portable manner.  In combination with
strongly portable driver code, it is hoped that any choice of platform
will have little impact on later work.

\subsection{General Functionality}

One major goal is making the CC traffic appear as similar as
possible to other traffic on the wire at present.  To address this,
the software modifies and retransmits captured network traffic.

Specifically, the transmitting software listens on the wire for an
Internet protocol version 4 (IPv4) packet of the necessary size to
encode a data byte that has not been marked as a covert packet (as
described later.)  The software then sets the header flag that
signifies the packet is part of a fragment chain, marks the packet as
a CC participant (excluding it from redundant reuse later), encodes a
data byte, recalculates the IP header checksum, and retransmits the
packet onto the wire.

The receiving software examines all packets that are available on the
wire, looking for packets that are part of the CC.  When one is found,
the encoded byte is written to the specified output.

\subsection{Packet Markers}

A means had to be found so that a packet containing covert data could
be distinguished from others in the traffic stream.  At the same time,
this means had to be virtually undetectable by a firewall, IDS, or
IPS.  Because the majority of these systems work by looking for a
specific signature or series of bits in the packet, the POC1 uses an
eight-bit identifier supplied by the user and shared between both
source and destination.  It is presumed that this identifier will
change periodically between one time the software is executed and
another.

\begin{figure}
\begin{center}
\begin{verbatim}
 0                   1                   2                   3   
 0 1 2 3 4 5 6 7 8 9 0 1 2 3 4 5 6 7 8 9 0 1 2 3 4 5 6 7 8 9 0 1 
+-+-+-+-+-+-+-+-+-+-+-+-+-+-+-+-+-+-+-+-+-+-+-+-+-+-+-+-+-+-+-+-+
|Version|  IHL  |Type of Service|          Total Length         |
|       |       |               | [Length minus covert payload] |
+-+-+-+-+-+-+-+-+-+-+-+-+-+-+-+-+-+-+-+-+-+-+-+-+-+-+-+-+-+-+-+-+
|         Identification        |     |                         |
|[Random Number]|[Pseudokey XOR]|Flags|      Fragment Offset    |
|   [Potential Payload Area]    |     |                         |
+-+-+-+-+-+-+-+-+-+-+-+-+-+-+-+-+-+-+-+-+-+-+-+-+-+-+-+-+-+-+-+-+
|  Time to Live |    Protocol   |         Header Checksum       |
+-+-+-+-+-+-+-+-+-+-+-+-+-+-+-+-+-+-+-+-+-+-+-+-+-+-+-+-+-+-+-+-+
|                        Source Address                         |
+-+-+-+-+-+-+-+-+-+-+-+-+-+-+-+-+-+-+-+-+-+-+-+-+-+-+-+-+-+-+-+-+
|                     Destination Address                       |
+-+-+-+-+-+-+-+-+-+-+-+-+-+-+-+-+-+-+-+-+-+-+-+-+-+-+-+-+-+-+-+-+
|                    Options                    |    Padding    |
+-+-+-+-+-+-+-+-+-+-+-+-+-+-+-+-+-+-+-+-+-+-+-+-+-+-+-+-+-+-+-+-+
\end{verbatim}
\caption{IP Header Fields from RFC791}
\label{IPHeader}
\end{center}
\end{figure}

To mark a packet as being involved with the covert channel, an 8-bit
random number is chosen for each packet to be sent, and is stored in
the bits zero through seven of the IP packet (see Figure 
\ref{IPHeader}) ``identification'' field.  The same random number is then
XOR'ed with the user-supplied identifier and the final value is stored
in bits eight through fifteen of the identification field.

The receiving software performs roughly the reverse to determine if a
particular packet is part of the covert channel.  The low eight bits
of the IP identification field are XOR'ed with the high eight bits,
then compared against the user-supplied key.  If they match, the
packet is processed further.  Otherwise it is discarded.

Use of the relatively small eight-bit identification marker results in
a one in 255 chance that a random packet will be falsely recognized as
part of the covert channel.  On a moderately busy network, this
creates a ridiculously high error rate that is unacceptable for
practical use.  This problem is addressed in the second proof of concept.

Additionally, the lack of stronger authentication in the packet leaves
the system open for man-in-the-middle attacks.  Defense against
this problem is out of the scope of the investigation.

\subsection{Data Encoding}

To meet dual needs of data transmission and stealth, the byte sent as
payload is encoded in a combination of the ``fragmentation offset'' and
the ``total length'' fields, as in Figure \ref{IPHeader}.  To provide a
light layer of obfuscation, the fragmentation offset is set by the
sender to the packet length minus the data byte for encoding. The
covert receiver reverses the process by subtracting the value of the
length field from the header offset to yield one data byte.

In the proof of concept programs, it is assumed that both the sender
and receiver are on the same subnet so no changes are made to the
destination IP address.  The sender transmits a modified packet and
assumes that the receiver will be able to detect and decode it.  The
``more fragments'' bit is set so that even if an unintentional
recipient receives the packet, it will be eventually timeout discarded
as part of a incomplete fragment chain.  The timeout causes the return
of an ``ICMP fragmentation timeout exceeded,'' message which may be a
path for later channel detection.

\section{Second Proof of Concept}

The functionality of the second proof of concept (POC2) is very
similar to POC1, with the major change being a move away from use of a
fragmentation offset greater than zero.  This allows additional fields
in the TCP header to be available without raising undue attention.
Additionally, a rudimentary metamorphic system was used so that data
was encoded in one of several different locations in the packet.

\subsection{Packet Marking}

As mentioned above, one of the major problems with the initial proof
of concept was false recognition of packets.  The eight-bit marker
allows an unacceptably high false positive rate.  A sample on UT's
wireless network yielded an unacceptable 34 false positives in a seven
minute period during the highest load time of the day.  For
simplicity, and to avoid requiring a two-way channel, POC2 uses 32-bit
identifiers in the TCP sequence number field.

Similar to POC1, a pseudo-key is provided by the user for stream
identification.  In POC2, the pseudo-key is 32-bits in length creating
a theoretical 1 in $2^{32}$ chance of a false positive.  The same
seven-minute sample with POC2 yielded zero false positives.  While
overall reliability could be improved by use of a rotating subset of a
longer pseudo-key, this is left for future investigation.

\begin{figure}
\begin{center}
\begin{verbatim}
 0                   1                   2                   3   
 0 1 2 3 4 5 6 7 8 9 0 1 2 3 4 5 6 7 8 9 0 1 2 3 4 5 6 7 8 9 0 1 
+-+-+-+-+-+-+-+-+-+-+-+-+-+-+-+-+-+-+-+-+-+-+-+-+-+-+-+-+-+-+-+-+
|          Source Port          |       Destination Port        |
+-+-+-+-+-+-+-+-+-+-+-+-+-+-+-+-+-+-+-+-+-+-+-+-+-+-+-+-+-+-+-+-+
|                        Sequence Number                        |
|                    [32-bit random number]                     |
+-+-+-+-+-+-+-+-+-+-+-+-+-+-+-+-+-+-+-+-+-+-+-+-+-+-+-+-+-+-+-+-+
|                    Acknowledgment Number                      |
|         [XOR of random number and 32-bit pseudo key]          |
+-+-+-+-+-+-+-+-+-+-+-+-+-+-+-+-+-+-+-+-+-+-+-+-+-+-+-+-+-+-+-+-+
|  Data |           |U|A|P|R|S|F|            Window             |
| Offset| Reserved  |R|C|S|S|Y|I|                               |
|       |           |G|K|H|T|N|N|    [Potential Payload Area]   |
+-+-+-+-+-+-+-+-+-+-+-+-+-+-+-+-+-+-+-+-+-+-+-+-+-+-+-+-+-+-+-+-+
|           Checksum            |         Urgent Pointer        |
|                               |    [Potential Payload Area]   |
+-+-+-+-+-+-+-+-+-+-+-+-+-+-+-+-+-+-+-+-+-+-+-+-+-+-+-+-+-+-+-+-+
|                    Options                    |    Padding    |
+-+-+-+-+-+-+-+-+-+-+-+-+-+-+-+-+-+-+-+-+-+-+-+-+-+-+-+-+-+-+-+-+
|                             data                              |
+-+-+-+-+-+-+-+-+-+-+-+-+-+-+-+-+-+-+-+-+-+-+-+-+-+-+-+-+-+-+-+-+
\end{verbatim}
\caption{TCP Header Fields from RFC793}
\label{TCPHeader}
\end{center}
\end{figure}

For each packet, a new 32-bit random number is stored in the TCP
``sequence number'' field.  The bitwise XOR of that random number and the
pseudo-key is stored in the TCP ``acknowledgment'' field, as seen in
Figure \ref{TCPHeader}.

To detect if an incoming packet is part of the CC, the receiving
software examines each packet looking for a TCP header.  If the XOR of
the TCP ``sequence number'' and the TCP ``acknowledgement'' fields
match the 32-bit pseudokey, the packet is decoded as being part of the
covert channel.

\subsection{Data Encoding}

In order for the TCP header to be an expected part of a packet
fragment, it must be seen in the beginning of the first fragment.
This means that the fragmentation offset must always be zero.  Because
the data encoding method used in POC1 requires a difference in 

Instead of encoding the payload data in the fragmentation offset, one
of three header field locations (IP Identification, TCP Window, and
TCP Urgent Pointer) as noted by ``\texttt{[Potential Payload Area]}'' in
Figures \ref{IPHeader} and \ref{TCPHeader}.  Which of these locations
is selected is based on the modulus of the 32-bit random number that
is stored in the TCP ``sequence number'' field.

\section{Detection and Defense}

Once the proof-of-concepts were debugged and deemed essentially
complete, their design was frozen so that the detection and
analysis findings would have a stationary target to address.

As was expected, default installations of \textit{Snort}--a popular
open-source IDS tool--were unable to recognize the CC traffic
specifically as a covert channel.  Presumably this is because 
no signature existed.  However, it was occasionally flagged as
anomalous data.
% TODO: really test the software against Snort

\subsection{Detecting this Covert Channel}
\label{Detection}

Multiple potential detection techniques were observed, based on the
final output from the covert channel software.  Unfortunately, few of
them lent themselves to the packet-at-a-time or stream reconstruction
analysis methods that our IDSs and IPSs provided.  Those that
showed signatures yielded a large number of false positives.  A
scoring system that suggests investigation of the hosts with the
largest number of signature hits.

\begin{itemize}
\item \textbf{Duplicate Packets:} The covert channel traffic modifies and
retransmits packets shortly after the originals are sent.  While it
would be difficult and unwise to block duplicate packets, the large
number of duplicate packets created by the CC transmitter would likely
be sufficient to merit investigation with the expectation of fouled
equipment or software.

\item \textbf{Unusual Field Values:} Some fields used to encode data
may have unusual values.  For example, the TCP Urgent Pointer may
point to an area past the end of the packet or the window size may not
have a power-of-two-aligned value.  While these are unusual, they're
very difficult to identify programmatically in such a way that doesn't
create an unusual number of false positives.

\item \textbf{Packet Preambles in Fragments:} False fragments
in POC1 will often appear to come from the middle of a large packet.
By using packets off the wire, data and headers that will commonly
come from the beginning of a packet, will now be appearing in the
middle of a larger packet.  Detection of this anomaly will likely only
occur due to visual packet inspection.

\item \textbf{Unusual-Length Fragments:} Packets are fragmented when
they are sent across a medium whose maximum transmission unit (MTU)
exceeds the length of the packet.  Ethernet provides the most common
MTU size: 1500 bytes.  Both POC1 and POC2 will create fragments that
are unusually small: sometimes approaching the \textit{minimum} packet
size.

\item \textbf{Unusual Protocol Fragments:} Both POC's use
fragmentation to avoid notice by upper network layers of the
destination system. While UDP fragments are reasonably commonplace
(especially on a network that uses Solaris-based NFS servers), TCP
goes out of its way (via \textit{Path MTU Discovery} documented in
RFC1191) to create packets that are smaller than the smallest MTU on
the path between sender and recipient.  Additionally, ICMP messages
are often small (read: less than any reasonable MTU) and any
fragmentation of these packets could be considered an anomaly.

\item \textbf{Unusual Order for TCP Sequence Numbers:} POC2 rewrites
the TCP sequence and acknowledgment numbers to mark packets.  Tools
such as
\textit{WireShark}\footnote{\texttt{http://www.wireshark.org/}} and
\textit{tcpdump}\footnote{\texttt{http://www.tcpdump.org/}} display
captured packets with relative sequence numbers based on the SYN
packets or the first TCP packet seen.  They flag any packets that are
outside the transmission window as either a lost segment or a
retransmission.

\item \textbf{Excessive Fragmentation Count:} Again, both POC's use
fragmentation to avoid notice by upper network layers of the
destination system.  This causes what can be an unusual number of
fragments (depending on the network) and an unusual amount of ICMP
Fragmentation Timeout backscatter.  Either of these can be counted and
the top hosts monitored or investigated.
\end{itemize}

\subsection{Defense Methods}

A proactive means for defending against both POC's is use of a
firewall that rewrites IP identification numbers and TCP sequence
numbers.  This firewall would use a bijective function to map the
original identification values on one side to a different value on the
other.  It would need to modify both incoming and outgoing packets in
order to completely remove the protected area from participation in
this type of channel.

Additionally, network monitoring for systems that show an unusually
large number of the traits mentioned in Section \ref{Detection} would
work as a reactive measure.

\subsection{Future Work}

In the process of examining the subject area, writing the proof of
concept software, and determining defenses against it, several areas
for future work were suggested.  These were not pursued due to
scheduling issues, financial resources, or scope limitations.

The most important area for further development appears to be
modification of the design to allow two-way communication or (at
least) acknowledgment responses.  Without that, there is no way to
guarantee reliable communication. Unfortunately, the present scheme
doesn't easily permit this without violating some design goals and
making the overall channel significantly easier to detect.

\begin{itemize}
\item \textbf{Covert Network Layer:} This scheme could be made more
valuable by enhancing the POC software to allow the use of the covert
channel as part of an existing network layer.  Either by providing a
network layer interface similar to IP, or as a higher layer interface
with sockets.  This would allow its use with many existing software
packages.

\item \textbf{Responses through ICMP Backscatter:}  One novel means of
implementing two-way communication would be to make use of the padding
or payload in the ICMP Fragmentation Timeout backscatter created as a
side-effect to the use of fragmentation packets.  This could be used
to send acknowledgments and increase the reliability of the channel,
even if full two-way communication is impossible.

\item \textbf{Encryption:} Incorporation of encryption could seriously
strengthen the usefulness of this type of covert channel.  Instead of
using random numbers and a pseudokey, use of a one-time pad or a
similar technique of evenly-distributed values could make the payload
more difficult to detect and decode by an observer.  Additionally, the
extreme vulnerability of this channel (once detected) to
man-in-the-middle insertion attacks could be largely mitigated through
the aforementioned technique.

\item \textbf{Erasure Codes:} Erasure codes create a larger message
based on an original such that the original message can be recreated
given a sufficiently-sized subset of the new encoding.  By encoding
the original message in this manner, a one-way stream can be created
that would strongly increase the chances that the message would arrive
unmodified.
\end{itemize}

% TODO: test software again to ensure it really works

\section{Conclusions}

Both covert channel proof-of-concept models met all the design goals
and constraints.

\begin{enumerate}
\item They were able to transfer data between the source and
destination without raising IDS or IPS alerts.
\item They did not interfere with the routing or delivery fields of
the packets that were modified.
\item There were no single or combination static data signatures unique to
the covert channels.
\item The channels used only serial-number and identification fields.
\item The channels used network traffic in a least-impact manner.
\end{enumerate}

% TODO: conclusion?

\end{document}
